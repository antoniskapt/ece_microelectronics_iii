\begin{table}[h]
	\begin{center}
		\begin{tabular}{|c|c|c|c|}
			\specialrule{1.25pt}{0pt}{0pt}
			\textbf{Σήμα}       & \textbf{Τιμές προσομοίωσης}    & \textbf{Τιμές εργαστηρίου}     \\\hline\hline
			$v_1$              & $14.9\unit{\volt}_\mathrm{pp}$ & $14.8\unit{\volt}_\mathrm{pp}$ \\\hline
			$v_2$                & $28.8\unit{\volt}_\mathrm{pp}$ & $28.0\unit{\volt}_\mathrm{pp}$ \\\hline
			$v_{\mathrm{out}}$  & $7.1\unit{\volt}_\mathrm{pp}$ & $-\unit{\volt}_\mathrm{pp}$ \\\hline %\\\specialrule{1.25pt}{0pt}{0pt}
		\end{tabular}
		\caption{Αποτελέσματα προσομοίωσης και εργαστηριακής εφαρμογής.}
		\label{table:ask2:conclusion}
	\end{center}
\end{table}

Από τα αποτελέσματα του πίνακα \ref{table:ask2:conclusion} είναι εμφανής η ομοιότητα των εργαστηριακών αποτελεσμάτων για τις τιμές $v_1$ και $v_2$ (διορθωμένη). Για την τιμή $v_{\mathrm{out}}$ δεν υπάρχει τιμή από την εργαστηριακή εφαρμογή, λόγω λανθασμένης κυματομορφής εξόδου που δεν επιλύθηκε. Επίσης οι περίοδοι των κυματομορφών προσομοίωσης και εργαστηρίου είναι παρόμοιες. Για την $v_1$ είναι περίπου $8\unit{\milli\second}$ και $5.5\unit{\milli\second}$ αντίστοιχα, ενώ για την $v_2$ προσεγγίζουν και οι δύο τα $20\unit{\milli\second}$.\par
Παρόμοια με την πρώτη άσκηση οι διαφορές μεταξύ των κυκλωμάτων μπορεί να οφείλονται στην ακρίβεια των μοντέλων της προσομοίωσης, στην ακρίβεια των μετρήσεων στο εργαστήριο ή και στην ακρίβεια των τιμών των εξαρτημάτων. Το πρόβλημα στην δημιουργία της κυματομορφής εξόδου στην εργαστηριακή εφαρμογή πιθανώς να οφείλεται σε λανθασμένη συνδεσμολογία ή κάποιο δυσλειτουργικό εξάρτημα του κυκλώματος.\par
