\begin{table}[h]
	\begin{center}
		\begin{tabular}{|c|c|c|c|}
			\specialrule{1.25pt}{0pt}{0pt}
			\textbf{Σήμα}      & \textbf{Θεωρητικές τιμές}       & \textbf{Τιμές προσομοίωσης}    & \textbf{Τιμές εργαστηρίου}     \\\hline\hline
			$v_1$              & $13.7\unit{\volt}_\mathrm{pp}$  & $14.0\unit{\volt}_\mathrm{pp}$ & $13.8\unit{\volt}_\mathrm{pp}$ \\\hline
			$v_2$              & $16.6\unit{\volt}_\mathrm{pp}$  & $16.1\unit{\volt}_\mathrm{pp}$ & $17.0\unit{\volt}_\mathrm{pp}$ \\\hline
			$v_{\mathrm{out}}$ & $11.66\unit{\volt}_\mathrm{pp}$ & $13.6\unit{\volt}_\mathrm{pp}$ & $13.2\unit{\volt}_\mathrm{pp}$ \\\hline
			$T_\mathrm{out}$   & $528\unit{\micro\second}$       & $670.7\unit{\micro\second}$      & $592\unit{\micro\second}$      \\\specialrule{1.25pt}{0pt}{0pt}
		\end{tabular}
		\caption{Αποτελέσματα θεωρητικής ανάλυσης, προσομοίωσης και εργαστηριακής εφαρμογής.}
		\label{table:ask1:conclusion}
	\end{center}
\end{table}

Όλα σχεδόν τα αποτελέσματα του πίνακα \ref{table:ask1:conclusion} είναι αρκετά κοντά μεταξύ τους. Μεγάλη απόκλιση παρουσιάζουν οι τιμές της περιόδου της $v_\mathrm{out}$. Συγκεκριμένα, η θεωρητική τιμή είναι κατά $64\unit{\micro\second}$ μικρότερη της πειραματικής και κατά $142.7\unit{\micro\second}$ μικρότερη της προσομοίωσης.\par
Επιπλέον, στο εργαστήριο η τιμή του ροοστάτη $R_v$, στη μέγιστη συχνότητα λειτουργίας, βρέθηκε στα $6.9\kohm$ ή $R=1\kohm+6.9\kohm=7.9\kohm$ και στην αντίστοιχη προσομοίωση η $R_v$ προέκυψε $5.6\kohm$ ή $R=1\kohm+5.6\kohm=6.5\kohm$. Οι τιμές είναι ικανοποιητικά κοντά, ιδίως λαμβάνοντας υπόψιν τις υπόλοιπες διαφορές στις τιμές των δύο κυκλωμάτων αλλά και τη δυσκολία της ρύθμισης του ροοστάτη στο εργαστήριο με ακρίβεια.\par
Οι διαφορές μεταξύ των κυκλωμάτων μπορεί να οφείλονται στην ακρίβεια των μοντέλων της προσομοίωσης, στην ακρίβεια των μετρήσεων στο εργαστήριο ή και στην ακρίβεια των τιμών των εξαρτημάτων. Τέλος, σημαντικό ρόλο παίζουν και οι συνθήκες περιβάλλοντος αλλά και τα τυχαία σφάλματα.\par