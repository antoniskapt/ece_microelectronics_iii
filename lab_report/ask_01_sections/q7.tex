Σύμφωνα με την σχέση \eqref{eq:ask1_vout} το πλάτος του τριγωνικού παλμού μπορεί να ρυθμιστεί μεταβάλλοντας την κλίση του. Αυτό είναι δυνατό αν τροποποιήσουμε τις μεταβλητές $R_f$ και $R_1$. Λαμβάνοντας ωστόσο υπόψιν την σχέση \eqref{eq:ask1_freq} οι μεταβλητές αυτές θα μεταβάλλουν την συχνότητα. Για να ρυθμιστεί το πλάτος χωρίς να αλλάξει η συχνότητα, πρέπει να αλλάξει το όριο της $v_2$, το οποίο εξαρτάται από τις διόδους Zener και ισούται με $-\(V_D+V_Z\)$. Οπότε χρησιμοποιώντας διόδους με διαφορετικές τάσεις Zener και threshold θα μπορέσουν να ρυθμίσουν το πλάτος του παλμού, δίχως να μεταβληθεί η συχνότητα.