
Οι προδιαγραφές του ζητούμενου τελεστικού ενισχυτή δίδονται στον πίνακα \ref{table:specs}.

\begin{table}[H]
	\begin{center}
		\begin{tabular}{|r|l|}
			\hline
			\multirow[|r|]{2}{*}{\textbf{Τροφοδοσία}}           & $V_{SS} =-2.31\unit{\volt}$                         \\\cline{2-2}
			                                                    & $V_{DD} =+2.31\unit{\volt}$                         \\\hline
			\multirow[|r|]{2}{*}{\textbf{Εύρος τάσεως εισόδου}} & $\min{\(v_{\mathrm{in}}\)} =-100\unit{\milli\volt}$ \\\cline{2-2}
			                                                    & $\max{\(v_{\mathrm{in}}\)} =+100\unit{\milli\volt}$ \\\hline
			\textbf{Κέρδος τάσεως}                              & $A_v >20.17\unit{\decibel}$                         \\\hline
			\textbf{Καταναλισκόμενη ισχύς}                      & $P_\mathrm{diss} <50.17\unit{\milli\watt}$          \\\hline
			\textbf{Gain bandwidth}                             & $\GB >7.17\unit{\mega\hertz}$                       \\\hline
			\textbf{Slew rate}                                  & $\SR >18.17\,\unit{\volt\per{\micro\sec}}$          \\\hline
			\textbf{Χωρητικότητα φορτίου}                       & $C_L =2.17\unit{\pico\farad}$                       \\\hline
		\end{tabular}
		\caption{Προδιαγραφές και δεδομένα για τον τελεστικό ενισχυτή του κυκλώματος \ref{circ:op_amp_schematic}.}
		\label{table:specs}
	\end{center}
\end{table}